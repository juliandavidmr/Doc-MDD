\documentclass[12pt,oneside,a4paper]{book}
\usepackage[latin1]{inputenc}
\usepackage[english, spanish]{babel}	%For languages characters and hyphenation
\usepackage{amsmath}
\usepackage{amsfonts}
\usepackage{titlesec}
\usepackage{amssymb}
\usepackage{graphicx}
\usepackage{array}
\usepackage{cite}
\usepackage[all]{xy}
\usepackage{fancyhdr}
\usepackage{lastpage}
\usepackage{caption}
\usepackage{subcaption}
\usepackage{multicol}
\usepackage{hhline}
\usepackage{multirow}
\usepackage{microtype}		%Makes pdf look better.
\usepackage[left=3.00cm, right=3.00cm, top=2.00cm, bottom=2.00cm]{geometry}
\author{Julian David Mora Ramos}
\title{Proyecto, implementaci�n de SIGEPI}
\usepackage[pdftex,
	pdfauthor={JULIAN DAVID MR},
	pdftitle={IMPLEMENTACI�N DEL SISTEMA SIGEPI},
	pdfsubject={DESARROLLO DE SOFTWARE},
	pdfkeywords={SOFTWARE, MDD, MDE, UDLA, DOCUMENTACI�N, DESARROLLO},
	pdfproducer={Universidad de la Amazonia},
	pdfcreator={@anlijudavid}]{hyperref}

% -------------------------
\setcounter{tocdepth}{4}
\setcounter{secnumdepth}{4}
\newcommand{\myparagraph}[1]{\paragraph{#1}\mbox{}\\}	
% -------------------------

% ----------------Tama�o de texto Chapters-----------------
\newcommand{\chapfnt}{\fontsize{16}{19}}
\newcommand{\secfnt}{\fontsize{14}{17}}
\newcommand{\ssecfnt}{\fontsize{12}{14}}

\titleformat{\chapter}[display]
{\normalfont\chapfnt\bfseries}{\chaptertitlename\ \thechapter}{20pt}{\chapfnt}

\titleformat{\section}
{\normalfont\secfnt\bfseries}{\thesection}{1em}{}

\titleformat{\subsection}
{\normalfont\ssecfnt\bfseries}{\thesubsection}{1em}{}

\titlespacing*{\chapter} {0pt}{50pt}{40pt}
\titlespacing*{\section} {0pt}{3.5ex plus 1ex minus .2ex}{2.3ex plus .2ex}
\titlespacing*{\subsection} {0pt}{3.25ex plus 1ex minus .2ex}{1.5ex plus .2ex}
% --------------------------------------------------------

% -----------------Config de grafica MVC------------------
\usepackage{tikz}
\usetikzlibrary{arrows,positioning} 
\tikzset{
	%Define standard arrow tip
	>=stealth',
	%Define style for boxes
	punkt/.style={
		rectangle,
		rounded corners,
		draw=black, very thick,
		text width=6.5em,
		minimum height=2em,
		text centered},
	% Define arrow style
	pil/.style={
		->,
		thick,
		shorten <=2pt,
		shorten >=2pt,}
}
% --------------------------------------------------------

% ------------Permite renderizar muchas tablas------------
\maxdeadcycles=1000
% --------------------------------------------------------

\begin{document}
	\markboth{"SIGEPI"}{''Universidad de la Amazonia''}
	
	% Paginas iniciales
	\title{%
	Implementaci�n del sistema SIGEPI usando herramientas de desarrollo dirigido por modelos (DSL) \\
	\large Implementaci�n de la plataforma web para la gesti�n de los proyectos presentados por los grupos y semilleros de investigaci�n en la Universidad de la Amazonia \\
}
\author{
	Julian David Mora Ramos $^{1}$
	\\
	\small{$^{1}$Universidad de la Amazonia, $^{1}$Florencia,Caquet�}\\
}
\maketitle
	\thispagestyle{empty}

\includegraphics{./resources/logoudla2.PNG}
\vspace{2cm}

\begin{center}
	{\Large {\bf UNIVERSIDAD DE LA AMAZONIA}}
	\vspace{5mm}

 	{\Large {Ingenier�a de Sistemas}}
    \\
  	\vspace{34mm}
	{\large {\bf PROYECTO FIN DE PREGRADO}}
  	\vspace{10mm}
    \\
  	{\Large {{\Huge {				
  		IMPLEMENTACI\'ON DEL SISTEMA INFORMACI�N PARA GESTI�N DOCUMENTAL SIGEPI
	}} \\[1cm] }}
  	\vspace{2cm}
	{\large {
  		{\bf Autor}: Julian David Mora Ramos\\
  		{\bf Tutores}: Heriberto Fernando Vargas\\ 
				  	   Diana Carolina Chico\\
				  	   Diana Mar�a Espinosa\\
  	}}
	\vspace{10mm}
  	{\large {Curso acad�mico 2011/2017}}
  	\vspace{1cm}
\end{center}
	\include{capitulos/dedicacion}
	%------------------------------------------------------
%   AGRADECIMIENTOS
%------------------------------------------------------

\chapter*{Agradecimientos}
%\markboth{AGRADECIMIENTOS23}{AGRADECIMIENTOS} % encabezado 

�Muchas gracias a todos!

	\begin{flushright}

\thispagestyle{empty}
\begingroup

\hspace{5em}

\textsc{NOTA DE ACEPTACI�N}

\vspace{2em}

\rule[1.5em]{20em}{0.5pt} % L�nea para la fecha
\rule[1.5em]{20em}{0.5pt} % L�nea para la fecha
\rule[1.5em]{20em}{0.5pt} % L�nea para la fecha
\rule[1.5em]{20em}{0.5pt} % L�nea para la fecha
\rule[1.5em]{20em}{0.5pt} % L�nea para la fecha
\rule[1.5em]{20em}{0.5pt} % L�nea para la fecha

\vspace{5em}
\rule[1em]{20em}{0.5pt} % L�nea para la fecha

\textsc{FIRMA DEL PRESIDENTE DEL JURADO}


\vspace{5em}
\rule[1em]{20em}{0.5pt} % L�nea para la fecha

\textsc{FIRMA DEL JURADO}


\vspace{5em}
\rule[1em]{20em}{0.5pt} % L�nea para la fecha

\textsc{FIRMA DEL JURADO}

\endgroup
\vspace*{\fill}
\end{flushright}

	\include{capitulos/declaracion}
	\pagestyle{empty}

\begin{center}
	\section*{Resumen}
	
	
	\section*{Abstract}
\end{center}
	
	\tableofcontents
	
	\cleardoublepage
	\addcontentsline{toc}{chapter}{Lista de figuras} % para que aparezca en el indice de contenidos
	\listoffigures % indice de figuras
	
	\cleardoublepage
	\addcontentsline{toc}{chapter}{Lista de tablas} % para que aparezca en el indice de contenidos
	\listoftables % indice de tablas
	
	% \pagenumbering{arabic} % para empezar la numeraci�n con n�meros
	% Empieza la numeraci�n de las p�ginas	
	\mainmatter
		
	% Contenido
	\chapter{Introducci�n}

Actualmente, adaptarse a las necesidades del cliente resulta algunas 
veces en un problema que aumenta con el transcurso del tiempo, las 
reglas de negocio cambian y consigo tambi�n la construcci�n de sistemas
de informaci�n, de esta manera, el mundo moderno exige gran demanda tecnol�gica
y adaptarse a esta linea de constantes transformaciones resulta
en una tarea compleja. Por esto, gracias a varias investigaciones
realizadas en este campo, se han logrado avances en ciertas �reas
fundamentales del desarrollo de software que influyen en la puesta 
en marcha de proyectos.
\\

Muchos proyectos de tipo software est�n siendo solicitados por entidades
para satisfacer m�ltiples prop�sitos. La demanda es alta cuando se 
usan buenas practicas de desarrollo junto con herramientas orientadas al
rendimiento y eficiencia en el cumplimiento de tareas.
Los tiempos de ejecuci�n y elaboraci�n de proyectos se
reducen cuando herramientas DSL (Lenguaje de dominio especifico)
son implementadas, ofreciendo caracter�sticas de generaci�n de
c�digo fuente reutilizable y/o componentes modulares.
\\

El lenguaje de dominio especifico no es una tecnolog�a emergente,
las primeras apariciones fueron a mediados de los 80s. 
Actualmente existen derivaciones de DSL para diferentes aplicaciones,
cada una dise�ada para la ejecuci�n de funcionalidades especificas, 
tales como MDD (Desarrollo Dirigido por Modelos), MDA (Arquitectura Dirigida 
por Modelos) y MDE (Ingenier�a Orientada a Modelos), siendo estos 
un conjunto de recursos interoperables que permiten ser usados en el
an�lisis, modelado y construcci�n de proyectos software.
Por otro lado, han aumentado la cantidad de mecanismos para agilizar
realizaci�n de proyectos, generalmente estos siguen la misma filosof�a 
de las DSL, por ejemplo, se pueden encontrar las ORM (Mapeo de Objetos 
Relacionales) para el control de m�ltiples gestores de bases de datos desde
un solo lenguaje base.
Del mismo modo, los ingenieros, programadores de computadoras, 
matem�ticos, estad�sticos y dem�s labores afines, hacen uso de herramientas 
DSL a partir de programas de computadora 
y lenguajes de programaci�n tales como R (Lenguaje estad�stico), 
interpretes de expresiones regulares, SQL (lenguaje de consulta estructurada), 
entre otros. Como se puede observar, un dominio espec�fico 
tiene un vocabulario especializado para describir las cosas que son 
particulares a ese dominio.
\\

El sistema de gesti�n documental de proyectos de investigaci�n, SIGEPI, 
es una plataforma web robusta, creada gracias a la implementaci�n de 
herramientas especializadas en el modelado y generaci�n de c�digo fuente. 
En la primera secci�n de este documento se nombran todos aquellos conceptos 
que detallan los procesos que se siguieron para la obtenci�n del c�digo 
fuente a partir de los esquemas dados al aplicar MDD. Posteriormente, se 
explica la arquitectura de desarrollo que fue usada para la plataforma web, 
incluyendo cada una de las librer�as y marcos de trabajo que fueron usados.
	\section[DESCRIPCI�N GENERAL DEL PROYECTO]{DESCRIPCI�N GENERAL DEL PROYECTO}

\subsection[Planteamiento del problema]{Planteamiento del problema}

\subsubsection[Contexto]{Contexto}
En la b�squeda de mejores procesos que ayuden a la optimizaci�n 
y mejoramiento de la productividad en el desarrollo de software, 
nuevas metodolog�as y herramientas han emergido, consigo vienen 
diferentes maneras de aplicar la ingeniera en la realizaci�n de 
tareas complejas para tiempos relativamente cortos. Un ejemplo 
com�n es el modelado de requisitos, siguiendo el est�ndar UML 
se pueden obtener esquemas que permiten la visualizaci�n de 
cada proceso por separado desde diferentes puntos de vista; 
diagramas de casos de uso, clases, bloques, secuencia, componentes, etc.

Hoy en d�a, es com�n encontrar variedades herramientas de 
modelado UML que permiten la generaci�n de c�digo fuente, 
pero existe una limitaci�n que se puede plantear sobre 
la misma ideolog�a 
que mantiene este mismo est�ndar, dado que se deben seguir 
estrictamente una serie de normas. Existen casos en donde es 
necesario dise�ar y crear un sistema de informaci�n siguiendo un
conjunto de normas no dadas por un est�ndar existente, 
propiamente personalizadas y generalmente creadas desde cero a 
partir de una base abstracta obtenida desde los requisitos, es 
decir, crear una serie de reglas de modelado a partir de un grupo
de requisitos. Aqu� es donde entra en acci�n el dise�o dirigido 
por modelos, conocido por sus siglas MDD, un paradigma de ingenier�a
de software que permite la manipulaci�n de grandes cantidades de
requerimientos \cite{nunez2016enfoque}, todo esto siguiendo un conjunto 
de reglas de modelado de esquemas personalizados con base 
fundamental en los requisitos. De esta manera, cada proceso que se 
propone para la construcci�n del sistema va a seguir 
estrictamente ese conjunto de reglas individualizadas.

\subsubsection[Formulaci�n del problema]{Formulaci�n del problema}
Actualmente, desarrollar un sistema de informaci�n suele ser 



\subsection[Justificaci�n]{Justificaci�n}
El dise�o e implementaci�n de sistemas inform�ticos generalmente se 
efect�a siguiendo la manera tradicional, que consiste en tomar como 
base inicial uno o m�s entornos de desarrollo (conocido en el ingles 
como: Framework) y desde ah� se inicia la creaci�n y adecuaci�n de 
cada elemento que va a conformar el sistema, muchas veces esto se 
lleva a cabo manualmente, un ejemplo claro es cuando se usa la 
arquitectura MVC (Modelo, Vista y Controlador) de cualquier framework; 
se construyen individualmente los controladores, modelos y vistas. 
Esta operaci�n requiere de bastante tiempo dependiendo de la 
complejidad del proyecto, todo esto sin incluir el tiempo que se 
requiere desde un inicio para modelar los esquemas necesarios, como 
los diagramas UML. Lo anterior implica que el desarrollo del sistema 
debe mantener en constante supervisi�n para velar por el cumplimiento 
de los requisitos que fueron regidos en la planificaci�n inicial.

El ciclo de vida en el
desarrollo de un sistema de informaci�n (SI) est� dado por ocho
etapas, a continuaci�n se nombran en el orden de ejecuci�n:
Planificaci�n, an�lisis, dise�o, desarrollo,
implementaci�n, pruebas, instalaci�n, y mantenimiento. Con el uso de
herramientas DSL se reduce el tiempo de desarrollo de un SI, 
debido a la eficacia en el avance de las etapas que suelen tomar mas tiempo.
El elemento ''dise�o'' y ''desarrollo'' tienden a trabajar conjuntamente,
donde el primero est� ligado de forma rigurosa al segundo gracias a un
lenguaje intermediario, este ultimo conocido como lenguaje de dominio
especifico, o DSL.

Tal es el caso de la plataforma web SIGEPI (Sistema de informaci�n 
para la gesti�n de proyectos de investigaci�n), donde se opt� por 
la implementaci�n de la herramienta DSL Tools, un entorno de
desarrollo creado por la multinacional Microsoft, conocido 
como ''.NET''. Esta herramienta es una notaci�n del lenguaje de 
dominio especifico, ofrece un complejo Kit de desarrollo de 
software (SDK) que se integra totalmente con el entorno de 
desarrollo integrado, Visual Studio. Todo esto se adapta completamente con los lineamientos establecidos por Universidad 
de la Amazonia, ya que el sistema puede ser integrado simult�neamente con el sistema misional de la misma, Chair�.



\subsection[Objetivos]{Objetivos}
\subsubsection[Objetivo general]{Objetivo general}
Implementar el sistema de informaci�n para la gesti�n de los procesos de presentaci�n, evaluaci�n y seguimiento de los proyectos de investigaci�n en la Universidad de la Amazonia, mediante el tratamiento y aplicaci�n de los resultados obtenidos en el uso de la arquitectura de desarrollo dirigida por modelos, MDD.

\subsubsection[Objetivo especifico]{Objetivo especifico}
\begin{itemize}
	\item Procesar y generar resultados (c�digo fuente) a partir de los meta-modelos definidos por los esquemas dise�ados en la arquitectura MDD.
	\item Aplicar el c�digo fuente obtenido por el procesamiento de los meta-modelos en un entorno de desarrollo basado en la web, espec�ficamente la arquitectura de tres capas de .NET MVC.
	\item Realizar pruebas de rendimiento, usabilidad y seguridad para verificar el cumplimiento de los requisitos no funcionales referentes al mismo.
\end{itemize}

	\section[MARCO REFERENCIAL]{MARCO REFERENCIAL}

\subsection[Marco te�rico]{Marco te�rico}

\subsubsection[Desarrollo Dirigido por Modelos. Conceptos]{Desarrollo Dirigido por Modelos. Conceptos}
El desarrollo dirigido por modelos es un paradigma que resuelve
muchos inconvenientes en el desarrollo de software, la causa 
surgi� desde los inicios de la d�cada de los 60s, cuando se
introdujo el concepto de "crisis del software"
\cite{pons2010desarrollo}, originado por la complejidad y el
costo requerido por las necesidades del cliente.

\subsubsection[DSL, Lenguaje de dominio especifico]{Lenguaje de dominio especifico}
lalalalalalala

\subsubsection[DSL \& Meta modelos]{DSL \& Meta modelos}
lalalalalalalalalalalalalalalalalala

\subsubsection[Arquitectura MVC]{Arquitectura MVC}
olaololaolaolaolaola olaolaolaolaolaolao laolaolao laolaolaola


\subsubsection[Framework de desarrollo Ext.NET]{Framework de desarrollo Ext.NET}
dsdsdsdsdsdsdsdsdsdsdsdsdsdsdsdsdsdsdsdsdsdsdsdsdsdsdsdsdsdsdsd
	\chapter{Procesos para presentaci\'on de proyectos}

Vicerector\'ia de investigaciones de la Universidad de la Amazonia cuenta con una serie de procesos determinados para la propuesta y aceptaci�n de proyectos de investigaci�n, generalmente estos �ltimos provienen de docentes [ver figura \ref{proc_proy_inv_ini_doc}], grupos y/o semilleros de investigaci�n [ver figura \ref{proc_proy_grup_sem_inv}].

\includegraphics[width=\textwidth]{resources/procesos/PROPUESTO_INICIATIVA_FINAL.PNG} %
\captionof{figure}{Procesos para proyectos de investigaci�n iniciativa docente}
\label{proc_proy_inv_ini_doc}


\includegraphics[width=\textwidth]{resources/procesos/PROPUESTO_CONVOCATORIA_FINAL.PNG} %
\captionof{figure}{Procesos para proyectos de grupos y semilleros de investigaci�n}
\label{proc_proy_grup_sem_inv}
	\chapter{Base de datos}

Diccionario de datos 

\begin{table}[ht]
	\caption{Tabla Usuario}
	\label{labelTableUsuario}
	\begin{tabular}{ |l|l|l|l|l|l| }
		\hline
		Nombre & Tipo de dato & Valor Null & Primaria & For\'anea & Comentario \\ \hline
		UserIdUsuario & Cadena & Si & Si & No & \\ \hline
		RolFKIdRol & Entero & Si & No & Si & \\ \hline
		UserNombreUsuario & Cadena & Si & No & No & \\ \hline
		UserCorreo & Cadena & Si & No & No & \\ \hline
		UserNombres & Cadena & Si & No & No & \\ \hline
		UserApellidos & Cadena & Si & No & No & \\ \hline
	\end{tabular}
\end{table}

\begin{table}[ht]
	\caption{Tabla Semillero}
	\label{labelTableSemillero}
	\begin{tabular}{ |l|l|l|l|l|l| }
		\hline
		Nombre & Tipo de dato & Valor Null & Primaria & For\'anea & Comentario \\ \hline
		SmlrIdSemillero & Entero & Si & Si & No & \\ \hline 
		SmlrNombre & Cadena & Si & No & No & \\ \hline 
		SmlrDescripcion & Texto largo & Si & No & No & \\ \hline 
		SmlrRutaLogo & Texto largo & Si & No & No & \\ \hline 
		SmlrMision & Texto largo & Si & No & No & \\ \hline 
		SmlrVision & Texto largo & Si & No & No & \\ \hline 
		SmlrRuta & Cadena & Si & No & No & \\ \hline 
		SmlrSigla & Cadena & Si & No & No & \\ \hline 
		SmlrJustificacion & Texto largo & Si & No & No & \\ \hline 
		SmlrMetodologiaTrabajo & Texto largo & Si & No & No & \\ \hline 
		SmlrAreasTrabajo & Texto largo & Si & No & No & \\ \hline 	
	\end{tabular}
\end{table}

\begin{table}[ht]
	\caption{Tabla Integrantes Semillero}
	\label{labelTableIntegrantesSemillero}
	\begin{tabular}{ |l|l|l|l|l|l| }
		\hline
		Nombre & Tipo de dato & Valor Null & Primaria & For\'anea & Comentario \\ \hline
		IsmlIdIntegrante & Entero & Si & Si & No & \\ \hline 
		UserFKIdUsuario & Entero & Si & No & Si & \\ \hline 
		SmlrFKIdSemillero & Entero & Si & No & Si & \\ \hline 
		IsmlFecha & Fecha & Si & No & No & \\ \hline 
		IsmlEstado & Cadena & Si & No & No & \\ \hline	
	\end{tabular}
\end{table}

\begin{table}[ht]
	\caption{Tabla Rol}
	\label{labelTableRol}
	\begin{tabular}{ |l|l|l|l|l|l| }
		\hline
		Nombre & Tipo de dato & Valor Null & Primaria & For\'anea & Comentario \\ \hline
		RolIdRol & Entero & Si & Si & No & \\ \hline 
		RolNombreRol & Cadena & Si & No & No & \\ \hline	
	\end{tabular}
\end{table}

\begin{table}[ht]
	\caption{Tabla Permisos}
	\label{labelTablePermisos}
	\begin{tabular}{ |l|l|l|l|l|l| }
		\hline
		Nombre & Tipo de dato & Valor Null & Primaria & For\'anea & Comentario \\ \hline
		PrmsIdPermiso & Entero & Si & Si & No & \\ \hline 
		PrmsNombrePermiso & Cadena & Si & No & No & \\ \hline 
		PrmsIcono & Cadena & Si & No & No & \\ \hline 	
	\end{tabular}
\end{table}


\begin{table}[ht]
	\caption{Tabla Subpermisos}
	\label{labelTableSubpermisos}
	\begin{tabular}{ |l|l|l|l|l|l| }
		\hline
		Nombre & Tipo de dato & Valor Null & Primaria & For\'anea & Comentario \\ \hline
		SpmsIdSubpermiso & Entero & Si & Si & No & \\ \hline 
		SpmsNombreSubpermiso & Cadena & Si & No & No & \\ \hline 
		SpmsURL & Cadena & Si & No & No & \\ \hline 
		PrmsFKIdPermiso & Entero & Si & No & Si & \\ \hline 	
	\end{tabular}
\end{table}

\begin{table}[ht]
	\caption{Tabla Menú Usuario}
	\label{labelTableMenuUsuario}
	\begin{tabular}{ |l|l|l|l|l|l| }
		\hline
		Nombre & Tipo de dato & Valor Null & Primaria & For\'anea & Comentario \\ \hline
		RolFKIdRol & Entero & Si & Si & Si & \\ \hline 
		SpmsFKIdSubpermiso & Entero & Si & Si & Si & \\ \hline 
		MusrEstado & Cadena & Si & No & No & \\ \hline 	
	\end{tabular}
\end{table}

\begin{table}[ht]
	\caption{Tabla Repositorio}
	\label{labelTableRepositorio}
	\begin{tabular}{ |l|l|l|l|l|l| }
		\hline
		Nombre & Tipo de dato & Valor Null & Primaria & For\'anea & Comentario \\ \hline
		RepoIdRepositorio & Entero & Si & Si & No & \\ \hline 
		RepoNombre & Cadena & Si & No & No & \\ \hline 
		RepoDescripcion & Texto largo & Si & No & No & \\ \hline 
		RepoRuta & Texto largo & Si & No & No & \\ \hline 
		RepoEstado & Cadena & Si & No & No & \\ \hline 	
	\end{tabular}
\end{table}


\begin{table}[ht]
	\caption{Tabla Documento}
	\label{labelTableDocumento}
	\begin{tabular}{ |l|l|l|l|l|l| }
		\hline
		Nombre & Tipo de dato & Valor Null & Primaria & For\'anea & Comentario \\ \hline
		DocuIdDocumento & Entero & Si & Si & No & \\ \hline 
		DocuNombre & Cadena & Si & No & No & \\ \hline 
		DocuRuta & Texto largo & Si & No & No & \\ \hline 
		DocuFecha & Fecha & Si & No & No & \\ \hline 
		DocuEstado & Cadena & No & No & No & \\ \hline 
		UserFKIdUsuario & BI Entero & Si & No & Si & \\ \hline 
		RepoFKIdRepositorio & Entero & Si & No & Si & \\ \hline 	
	\end{tabular}
\end{table}

\begin{table}[ht]
	\caption{Tabla Correos Enviados}
	\label{labelTableCorreosEnviados}
	\begin{tabular}{ |l|l|l|l|l|l| }
		\hline
		Nombre & Tipo de dato & Valor Null & Primaria & For\'anea & Comentario \\ \hline
		CrnvIdCorreoEnviado & Entero & Si & Si & No & \\ \hline 
		CrnvTipo & Cadena & Si & No & No & \\ \hline 
		CrnvDescripcion & Texto largo & Si & No & No & \\ \hline 
		CrnvEstadoEnvio & VARCHAR(50) & Si & No & No & \\ \hline 
		CrnvFechaEnvio & Fecha & Si & No & No & \\ \hline 
		SmlrFKIdSemillero & Entero & Si & No & Si & \\ \hline 
		UserFKDestinatario & Cadena & Si & No & Si & \\ \hline 	
	\end{tabular}
\end{table}


\begin{table}[ht]
	\caption{Tabla Acceso Sistema}
	\label{labelTableAccesoSistema}
	\begin{tabular}{ |l|l|l|l|l|l| }
		\hline
		Nombre & Tipo de dato & Valor Null & Primaria & For\'anea & Comentario \\ \hline
		AstmIdAcceso & BI Entero & Si & Si & No & \\ \hline 
		AstmFecha & Fecha & Si & No & No & \\ \hline 
		AstmIP & Cadena & Si & No & No & \\ \hline 
		UserFKIdUsuario & BI Entero & Si & No & Si & \\ \hline 	
	\end{tabular}
\end{table}


\begin{table}[ht]
	\caption{Tabla Solicitudes}
	\label{labelTableSolicitudes}
	\begin{tabular}{ |l|l|l|l|l|l| }
		\hline
		Nombre & Tipo de dato & Valor Null & Primaria & For\'anea & Comentario \\ \hline
		SlctIdSolicitud & Entero & Si & Si & No & \\ \hline 
		CursIdCurso & Entero & Si & No & No & \\ \hline 
		UserFKEstudiante & BI Entero & Si & No & Si & \\ \hline 
		UserFKAdmin & BI Entero & Si & No & Si & \\ \hline 
		SlctFechaSolicitud & Fecha & Si & No & No & \\ \hline 
		SlctFechaRespuesta & Fecha & No & No & No & \\ \hline 
		SlctEstado & Cadena & No & No & No & \\ \hline 
		SlctComentario & Texto largo & Si & No & No & \\ \hline 	
	\end{tabular}
\end{table}


\begin{table}[ht]
	\caption{Tabla Grupo Investigaci\'on}
	\label{labelTableGrupoInvestigacion}
	\begin{tabular}{ |l|l|l|l|l|l| }
		\hline
		Nombre & Tipo de dato & Valor Null & Primaria & For\'anea & Comentario \\ \hline
		GrivIdGrupoInv & Entero & Si & Si & No & \\ \hline 
		GrivFechaCreacion & Fecha & Si & No & No & \\ \hline 
		GrivNombre & Cadena & Si & No & No & \\ \hline 
		GrivMision & Texto largo & Si & No & No & \\ \hline 
		GrivVision & Texto largo & Si & No & No & \\ \hline 	
	\end{tabular}
\end{table}


\begin{table}[ht]
	\caption{Tabla Programa}
	\label{labelTablePrograma}
	\begin{tabular}{ |l|l|l|l|l|l| }
		\hline
		Nombre & Tipo de dato & Valor Null & Primaria & For\'anea & Comentario \\ \hline
		ProgIdPrograma & Entero & Si & Si & No & \\ \hline 
		ProgNombre & Cadena & Si & No & No & \\ \hline 
		ProgCodigo & Cadena & No & No & No & \\ \hline 
		FacuFKIdFacultad & Entero & Si & No & Si & \\ \hline 	
	\end{tabular}
\end{table}


\begin{table}[ht]
	\caption{Tabla Programas Semillero}
	\label{labelTableProgramasSemillero}
	\begin{tabular}{ |l|l|l|l|l|l| }
		\hline
		Nombre & Tipo de dato & Valor Null & Primaria & For\'anea & Comentario \\ \hline
		ProgFKIdPrograma & Entero & Si & Si & Si & \\ \hline 
		SmlrFKIdSemillero & Entero & Si & Si & Si & \\ \hline 
		PgsmEstado & ENUM('T', 'F') & Si & No & No & \\ \hline 	
	\end{tabular}
\end{table}


\begin{table}[ht]
	\caption{Tabla Programas Grupos Inv}
	\label{labelTableProgramasGruposInv}
	\begin{tabular}{ |l|l|l|l|l|l| }
		\hline
		Nombre & Tipo de dato & Valor Null & Primaria & For\'anea & Comentario \\ \hline
		GrivFKIdGrupoInv & Entero & Si & Si & Si & \\ \hline 
		ProgFKIdPrograma & Entero & Si & Si & Si & \\ \hline 
		PggrEstado & ENUM('T', 'F') & Si & No & No & \\ \hline 
		PggrFecha & Fecha & Si & No & No & \\ \hline 
		PggrCartaAval & Texto largo & Si & No & No & \\ \hline 	
	\end{tabular}
\end{table}


\begin{table}[ht]
	\caption{Tabla Proyectos}
	\label{labelTableProyectos}
	\begin{tabular}{ |l|l|l|l|l|l| }
		\hline
		Nombre & Tipo de dato & Valor Null & Primaria & For\'anea & Comentario \\ \hline
		ProyIdProyecto & Entero & Si & Si & No & \\ \hline 
		ProyNombre & Texto largo & Si & No & No & \\ \hline 
		ProyDescripcion & Texto largo & Si & No & No & \\ \hline 
		ProyTipo & Cadena & Si & No & No & \\ \hline 
		ProSitado & Cadena & Si & No & No & \\ \hline 
		LinvFKIdLineaInv & Entero & Si & No & Si & \\ \hline 
		ConvFKIdConvocatoria & Entero & Si & No & Si & \\ \hline 
		GrivFKIdGrupoInv & Entero & Si & No & Si & \\ \hline 
		ProgFKIdPrograma & Entero & Si & No & Si & \\ \hline 
		ProyFormulacionProblema & Texto largo & Si & No & No & \\ \hline 
		ProyJustificacion & Texto largo & Si & No & No & \\ \hline 
		ProyMarcoTeorico & Texto largo & Si & No & No & \\ \hline 
		ProyMetodologia & Texto largo & Si & No & No & \\ \hline 
		ProyBibliografia & Texto largo & Si & No & No & \\ \hline 
		ProyAnexos & Texto largo & Si & No & No & \\ \hline 	
	\end{tabular}
\end{table}


\begin{table}[ht]
	\caption{Tabla Rol Semillero}
	\label{labelTableRolSemillero}
	\begin{tabular}{ |l|l|l|l|l|l| }
		\hline
		Nombre & Tipo de dato & Valor Null & Primaria & For\'anea & Comentario \\ \hline
		RsmlIdRolSemillero & Entero & Si & Si & No & \\ \hline 
		RsmlNombre & Cadena & Si & No & No & \\ \hline 
		RsmlEstado & Cadena & Si & No & No & \\ \hline 
		SmlrFKIdSemillero & Entero & Si & No & Si & \\ \hline 	
	\end{tabular}
\end{table}


\begin{table}[ht]
	\caption{Tabla Par Evaluador}
	\label{labelTableParEvaluador}
	\begin{tabular}{ |l|l|l|l|l|l| }
		\hline
		Nombre & Tipo de dato & Valor Null & Primaria & For\'anea & Comentario \\ \hline
		PaevIdParEvaluador & Entero & Si & Si & No & \\ \hline 
		PaevFecha & Fecha & Si & No & No & \\ \hline 
		PaevEstado & Cadena & Si & No & No & \\ \hline 
		PaevTipo & Cadena & Si & No & No & \\ \hline 
		UserFKParEvaluador & Cadena & Si & No & Si & \\ \hline 	
	\end{tabular}
\end{table}


\begin{table}[ht]
	\caption{Tabla Actividad Semillero}
	\label{labelTableActividadSemillero}
	\begin{tabular}{ |l|l|l|l|l|l| }
		\hline
		Nombre & Tipo de dato & Valor Null & Primaria & For\'anea & Comentario \\ \hline
		ActsIdActividad & Entero & Si & Si & No & \\ \hline 
		ActsDescripcion & Texto largo & Si & No & No & \\ \hline 
		ActsFechaCreacion & Fecha & Si & No & No & \\ \hline 
		ActsEstado & Cadena & Si & No & No & \\ \hline 
		ActsFechaInicio & DATE & Si & No & No & \\ \hline 
		ActsFechaFin & DATE & Si & No & No & \\ \hline 
		MetsFKIdMetaSemillero & Entero & Si & No & Si & \\ \hline 	
	\end{tabular}
\end{table}


\begin{table}[ht]
	\caption{Tabla Convocatoria}
	\label{labelTableConvocatoria}
	\begin{tabular}{ |l|l|l|l|l|l| }
		\hline
		Nombre & Tipo de dato & Valor Null & Primaria & For\'anea & Comentario \\ \hline
		ConvIdConvocatoria & Entero & Si & Si & No & \\ \hline 
		ConvNombre & Texto largo & Si & No & No & \\ \hline 
		ConvDescripcion & Texto largo & Si & No & No & \\ \hline 
		ConvObservaciones & Texto largo & No & No & No & \\ \hline 
		ConvFechaCreacion & Fecha & Si & No & No & \\ \hline 
		ConvFechaInicio & DATE & Si & No & No & \\ \hline 
		ConvFechaFin & DATE & Si & No & No & \\ \hline 
		ConvEstado & Cadena & Si & No & No & \\ \hline 
		TconFKIdTipoConvocatoria & Entero & Si & No & Si & \\ \hline 
		ConvOferente & Cadena & Si & No & No & \\ \hline 
		ConvDestinatario & Cadena & Si & No & No & \\ \hline 	
	\end{tabular}
\end{table}


\begin{table}[ht]
	\caption{Tabla Banco Preliminar}
	\label{labelTableBancopreliminar}
	\begin{tabular}{ |l|l|l|l|l|l| }
		\hline
		Nombre & Tipo de dato & Valor Null & Primaria & For\'anea & Comentario \\ \hline
		BpreIdBancoPreliminar & Entero & Si & Si & No & \\ \hline 
		BpreDescripcion & Cadena & Si & No & No & \\ \hline 
		BpreEstado & Cadena & Si & No & No & \\ \hline 
		ConvFKIdConvocatoria & Entero & Si & No & Si & \\ \hline 
		BpreFecha & Fecha & Si & No & No & \\ \hline 	
	\end{tabular}
\end{table}


\begin{table}[ht]
	\caption{Tabla Proyectos preliminares}
	\label{labelTableProyectospreliminares}
	\begin{tabular}{ |l|l|l|l|l|l| }
		\hline
		Nombre & Tipo de dato & Valor Null & Primaria & For\'anea & Comentario \\ \hline
		PpreIdProyectoPreliminar & Entero & Si & Si & No & \\ \hline 
		ProyFKIdProyecto & Entero & Si & No & Si & \\ \hline 
		BpreFKIdBancoPreliminar & Entero & Si & No & Si & \\ \hline 
		PpreEstado & Cadena & Si & No & No & \\ \hline 
		PpreFecha & Fecha & Si & No & No & \\ \hline 	
	\end{tabular}
\end{table}


\begin{table}[ht]
	\caption{Tabla Banco definitivo}
	\label{labelTableBancodefinitivo}
	\begin{tabular}{ |l|l|l|l|l|l| }
		\hline
		Nombre & Tipo de dato & Valor Null & Primaria & For\'anea & Comentario \\ \hline
		BdefIdBancoDefinitivo & Entero & Si & Si & No & \\ \hline 
		BdefEstado & Cadena & Si & No & No & \\ \hline 
		BdefFecha & Fecha & Si & No & No & \\ \hline 
		ConvFKIdConvocatoria & Entero & Si & No & Si & \\ \hline 	
	\end{tabular}
\end{table}


\begin{table}[ht]
	\caption{Tabla Proyectos definitivos}
	\label{labelTableProyectosdefinitivos}
	\begin{tabular}{ |l|l|l|l|l|l| }
		\hline
		Nombre & Tipo de dato & Valor Null & Primaria & For\'anea & Comentario \\ \hline
		PdefIdProyectoDefinitivo & Entero & Si & Si & No & \\ \hline 
		PdefEstado & Cadena & Si & No & No & \\ \hline 
		PdefFecha & Fecha & Si & No & No & \\ \hline 
		BdefFKIdBancoDefinitivo & Entero & Si & No & Si & \\ \hline 
		ProyFKIdProyecto & Entero & Si & No & Si & \\ \hline 	
	\end{tabular}
\end{table}


\begin{table}[ht]
	\caption{Tabla Evaluaci\'on}
	\label{labelTableEvaluacion}
	\begin{tabular}{ |l|l|l|l|l|l| }
		\hline
		Nombre & Tipo de dato & Valor Null & Primaria & For\'anea & Comentario \\ \hline
		EvlcIdEvaluacion & BI Entero & Si & Si & No & \\ \hline 
		EvlcDescripcion & Cadena & Si & No & No & \\ \hline 
		EvlcFecha & Fecha & Si & No & No & \\ \hline 
		EvlcEstado & Cadena & Si & No & No & \\ \hline 
		EvlcResultado & Cadena & Si & No & No & \\ \hline 
		AparFKIdAsignacion & Entero & Si & No & Si & \\ \hline 	
	\end{tabular}
\end{table}


\begin{table}[ht]
	\caption{Tabla Tipo Convocatoria}
	\label{labelTableTipoConvocatoria}
	\begin{tabular}{ |l|l|l|l|l|l| }
		\hline
		Nombre & Tipo de dato & Valor Null & Primaria & For\'anea & Comentario \\ \hline
		TconIdTipoConvocatoria & Entero & Si & Si & No & \\ \hline 
		TconNombre & Cadena & Si & No & No & \\ \hline 
		TconEstado & Cadena & Si & No & No & \\ \hline 	
	\end{tabular}
\end{table}


\begin{table}[ht]
	\caption{Tabla Linea Investigaci\'on}
	\label{labelTableLineaInvestigacion}
	\begin{tabular}{ |l|l|l|l|l|l| }
		\hline
		Nombre & Tipo de dato & Valor Null & Primaria & For\'anea & Comentario \\ \hline
		LinvIdLineaInv & Entero & Si & Si & No & \\ \hline 
		LinvNombreLinea & Cadena & Si & No & No & \\ \hline 
		LinvDescripcion & Texto largo & Si & No & No & \\ \hline 
		ProgFKIdPrograma & Entero & Si & No & Si & \\ \hline 	
	\end{tabular}
\end{table}


\begin{table}[ht]
	\caption{Tabla Facultad}
	\label{labelTableFacultad}
	\begin{tabular}{ |l|l|l|l|l|l| }
		\hline
		Nombre & Tipo de dato & Valor Null & Primaria & For\'anea & Comentario \\ \hline
		FacuIdFacultad & Entero & Si & Si & No & \\ \hline 
		FacuNombre & Cadena & Si & No & No & \\ \hline 
		FacuEstado & Cadena & Si & No & No & \\ \hline 	
	\end{tabular}
\end{table}


\begin{table}[ht]
	\caption{Tabla Evaluador Interno}
	\label{labelTableEvaluadorInterno}
	\begin{tabular}{ |l|l|l|l|l|l| }
		\hline
		Nombre & Tipo de dato & Valor Null & Primaria & For\'anea & Comentario \\ \hline
		EvaiIdEvaluadorInterno & Entero & Si & Si & No & \\ \hline 
		ProgFKIdPrograma & Entero & Si & No & Si & \\ \hline 
		PaevFKIdParEvaluador & Entero & Si & No & Si & \\ \hline 	
	\end{tabular}
\end{table}


\begin{table}[ht]
	\caption{Tabla Asignaci\'on Par Evaluaci\'on}
	\label{labelTableAsignacionParEvaluacion}
	\begin{tabular}{ |l|l|l|l|l|l| }
		\hline
		Nombre & Tipo de dato & Valor Null & Primaria & For\'anea & Comentario \\ \hline
		AparIdAsignacion & Entero & Si & Si & No & \\ \hline 
		ProyFKIdProyecto & Entero & Si & No & Si & \\ \hline 
		PaevFKIdParEvaluador & Entero & Si & No & Si & \\ \hline 
		AparEstado & Cadena & Si & No & No & \\ \hline 
		AparFecha & Fecha & Si & No & No & \\ \hline 
		UserFKAsignadoPor & Cadena & Si & No & Si & \\ \hline 	
	\end{tabular}
\end{table}


\begin{table}[ht]
	\caption{Tabla Preevaluacion}
	\label{labelTablePreevaluacion}
	\begin{tabular}{ |l|l|l|l|l|l| }
		\hline
		Nombre & Tipo de dato & Valor Null & Primaria & For\'anea & Comentario \\ \hline
		PrevIdPreevaluacion & Entero & Si & Si & No & \\ \hline 
		PrevDescripcion & Texto largo & Si & No & No & \\ \hline 
		PrevFecha & Fecha & Si & No & No & \\ \hline 
		PrevEstado & Cadena & Si & No & No & \\ \hline 
		PrevResultado & Cadena & Si & No & No & \\ \hline 
		InstFKIdInstancia & Entero & Si & No & Si & \\ \hline 
		ProyFKIdProyecto & Entero & Si & No & Si & \\ \hline 
		UserFKIdUsuario & Cadena & Si & No & Si & \\ \hline 	
	\end{tabular}
\end{table}


\begin{table}[ht]
	\caption{Tabla Instancia}
	\label{labelTableInstancia}
	\begin{tabular}{ |l|l|l|l|l|l| }
		\hline
		Nombre & Tipo de dato & Valor Null & Primaria & For\'anea & Comentario \\ \hline
		InstIdInstancia & Entero & Si & Si & No & \\ \hline 
		InstNombre & Cadena & Si & No & No & \\ \hline 
		InstDescripcion & Texto largo & Si & No & No & \\ \hline 
		InstEstado & Cadena & Si & No & No & \\ \hline 	
	\end{tabular}
\end{table}


\begin{table}[ht]
	\caption{Tabla Integrantes Grupo}
	\label{labelTableIntegrantesGrupo}
	\begin{tabular}{ |l|l|l|l|l|l| }
		\hline
		Nombre & Tipo de dato & Valor Null & Primaria & For\'anea & Comentario \\ \hline
		IgruIdIntegrante & Entero & Si & Si & No & \\ \hline 
		UserFKIdUsuario & Cadena & Si & No & Si & \\ \hline 
		GrivFKIdGrupoInv & Entero & Si & No & Si & \\ \hline 
		IgruEstado & Cadena & Si & No & No & \\ \hline 
		IgruFecha & Fecha & Si & No & No & \\ \hline 	
	\end{tabular}
\end{table}


\begin{table}[ht]
	\caption{Tabla Proyectos Semilleros}
	\label{labelTableProyectosSemilleros}
	\begin{tabular}{ |l|l|l|l|l|l| }
		\hline
		Nombre & Tipo de dato & Valor Null & Primaria & For\'anea & Comentario \\ \hline
		PrsmIdProyecto & Entero & Si & Si & No & \\ \hline 
		PrsmFecha & Fecha & Si & No & No & \\ \hline 
		PrsmEstado & Cadena & Si & No & No & \\ \hline 
		ProyFKIdProyecto & Entero & Si & No & Si & \\ \hline 
		SmlrFKIdSemillero & Entero & Si & No & Si & \\ \hline 	
	\end{tabular}
\end{table}


\begin{table}[ht]
	\caption{Tabla Proyectos Grupos}
	\label{labelTableProyectosGrupos}
	\begin{tabular}{ |l|l|l|l|l|l| }
		\hline
		Nombre & Tipo de dato & Valor Null & Primaria & For\'anea & Comentario \\ \hline
		PrgrIdProyecto & Entero & Si & Si & No & \\ \hline 
		PrgrFecha & Fecha & Si & No & No & \\ \hline 
		PrgrEstado & Cadena & Si & No & No & \\ \hline 
		ProyFKIdProyecto & Entero & Si & No & Si & \\ \hline 
		GrivFKIdGrupoInv & Entero & Si & No & Si & \\ \hline 
		PrgrLugarEjecucion & Cadena & Si & No & No & \\ \hline 
		MpioFKIdMpio & Entero & Si & No & Si & \\ \hline 
		PrgrDuracion & DOUBLE & Si & No & No & \\ \hline 
		TinvFKIdTipo & Entero & Si & No & Si & \\ \hline 
		PrgrValorTotal & BI Entero & Si & No & No & \\ \hline 
		PrgrValorSolicitado & BI Entero & Si & No & No & \\ \hline 
		PrgrValorContrapartida & BI Entero & Si & No & No & \\ \hline 
		PrgrTotalInvestigadores & Entero & Si & No & No & \\ \hline 
		PrgrTipo & Cadena & Si & No & No & \\ \hline 
		PrgrResultadosEsperados & Texto largo & Si & No & No & \\ \hline 
		PrgrEstraComunicacion & Texto largo & Si & No & No & \\ \hline 
		PrgrPalabrasClave & Texto largo & Si & No & No & \\ \hline 
		& Cadena & No & No & No & \\ \hline 	
	\end{tabular}
\end{table}


\begin{table}[ht]
	\caption{Tabla Roles Integrantes Semillero}
	\label{labelTableRolesIntegrantesSemillero}
	\begin{tabular}{ |l|l|l|l|l|l| }
		\hline
		Nombre & Tipo de dato & Valor Null & Primaria & For\'anea & Comentario \\ \hline
		RismIdRol & Entero & Si & Si & No & \\ \hline 
		RismFecha & Fecha & Si & No & No & \\ \hline 
		RismEstado & Cadena & Si & No & No & \\ \hline 
		IsmlFKIdIntegrante & Entero & Si & No & Si & \\ \hline 
		RsmlFKIdRolSemillero & Entero & Si & No & Si & \\ \hline 	
	\end{tabular}
\end{table}


\begin{table}[ht]
	\caption{Tabla Rol Grupo}
	\label{labelTableRolGrupo}
	\begin{tabular}{ |l|l|l|l|l|l| }
		\hline
		Nombre & Tipo de dato & Valor Null & Primaria & For\'anea & Comentario \\ \hline
		RgruIdRolGrupo & Entero & Si & Si & No & \\ \hline 
		RgruNombre & Cadena & Si & No & No & \\ \hline 
		RgruEstado & Cadena & Si & No & No & \\ \hline 
		GrivFKIdGrupoInv & Entero & Si & No & Si & \\ \hline 	
	\end{tabular}
\end{table}


\begin{table}[ht]
	\caption{Tabla Roles Integrantes Grupo}
	\label{labelTableRolesIntegrantesGrupo}
	\begin{tabular}{ |l|l|l|l|l|l| }
		\hline
		Nombre & Tipo de dato & Valor Null & Primaria & For\'anea & Comentario \\ \hline
		RigrIdRol & Entero & Si & Si & No & \\ \hline 
		RigrFecha & Fecha & Si & No & No & \\ \hline 
		RigrEstado & Cadena & Si & No & No & \\ \hline 
		IgruFKIdIntegrante & Entero & Si & No & Si  & \\ \hline 
		RgruFKIdRolGrupo & Entero & Si & No & Si & \\ \hline 	
	\end{tabular}
\end{table}


\begin{table}[ht]
	\caption{Tabla Objetivos Proyecto Semillero}
	\label{labelTableObjetivosProyectoSemillero}
	\begin{tabular}{ |l|l|l|l|l|l| }
		\hline
		Nombre & Tipo de dato & Valor Null & Primaria & For\'anea & Comentario \\ \hline
		ObpsIdObjetivo & Entero & Si & Si & No & \\ \hline 
		ObpsID & Cadena & Si & No & No & \\ \hline 
		ObpsDescripcion & Cadena & Si & No & No & \\ \hline 
		ObpsFecha & Fecha & Si & No & No & \\ \hline 
		ObpsEstado & Cadena & Si & No & No & \\ \hline 
		ObpsTipo & Cadena & Si & No & No & \\ \hline 
		PrsmFKIdProyecto & Entero & Si & No & Si & \\ \hline 	
	\end{tabular}
\end{table}


\begin{table}[ht]
	\caption{Tabla Informe Parcial}
	\label{labelTableInformeParcial}
	\begin{tabular}{ |l|l|l|l|l|l| }
		\hline
		Nombre & Tipo de dato & Valor Null & Primaria & For\'anea & Comentario \\ \hline
		InpaIdInformeParcial & Entero & Si & Si & No & \\ \hline 
		InpaDescripcion & Texto largo & Si & No & No & \\ \hline 
		InpaFecha & Fecha & Si & No & No & \\ \hline 
		InpaEstado & Cadena & Si & No & No & \\ \hline 
		InpaMetodologia & Texto largo & Si & No & No & \\ \hline 
		InpaResultados & Texto largo & Si & No & No & \\ \hline 
		InpaImpacto & Texto largo & Si & No & No & \\ \hline 
		InpaEstraDivulgacion & Texto largo & Si & No & No & \\ \hline 
		ProyFKIdProyecto & Entero & Si & No & Si & \\ \hline 
		InpaDocSoporte & Texto largo & Si & No & No & \\ \hline 
		InpaPresupuesto & Texto largo & Si & No & No & \\ \hline 	
	\end{tabular}
\end{table}


\begin{table}[ht]
	\caption{Tabla Integrantes Proyecto Semillero}
	\label{labelTableIntegrantesProyectoSemillero}
	\begin{tabular}{ |l|l|l|l|l|l| }
		\hline
		Nombre & Tipo de dato & Valor Null & Primaria & For\'anea & Comentario \\ \hline
		IprsIdIntegrante & Entero & Si & Si & No & \\ \hline 
		IprsFecha & Fecha & Si & No & No & \\ \hline 
		IprsEstado & Cadena & Si & No & No & \\ \hline 
		RolpFKIdRolProyecto & Entero & Si & No & Si & \\ \hline 
		IsmlFKIdIntegrante & Entero & Si & No & Si & \\ \hline 
		PrsmFKIdProyecto & Entero & Si & No & Si &  \\ \hline 	
	\end{tabular}
\end{table}


\begin{table}[ht]
	\caption{Tabla Rol Proyecto}
	\label{labelTableRolProyecto}
	\begin{tabular}{ |l|l|l|l|l|l| }
		\hline
		Nombre & Tipo de dato & Valor Null & Primaria & For\'anea & Comentario \\ \hline
		RolpIdRolProyecto & Entero & Si & Si & No & \\ \hline 
		RolpNombre & Cadena & Si & No & No & \\ \hline 
		RolpEstado & Cadena & Si & No & No & \\ \hline 	
	\end{tabular}
\end{table}


\begin{table}[ht]
	\caption{Tabla Informe Final}
	\label{labelTableInformeFinal}
	\begin{tabular}{ |l|l|l|l|l|l| }
		\hline
		Nombre & Tipo de dato & Valor Null & Primaria & For\'anea & Comentario \\ \hline
		InfiIdInformeFinal & Entero & Si & Si & No & \\ \hline 
		InfiProblemaPlanteado & Texto largo & Si & No & No & \\ \hline 
		InfiHipotesis & Texto largo & Si & No & No & \\ \hline 
		InfiAvanceObtenido & Texto largo & Si & No & No & \\ \hline 
		InfiRetosPlanteados & Texto largo & Si & No & No & \\ \hline 
		InfiGradoComprobacion & Texto largo & Si & No & No & \\ \hline 
		InfiGestionProyecto & Texto largo & Si & No & No & \\ \hline 
		InfiObservaciones & Texto largo & Si & No & No & \\ \hline 
		InfiRecomendaciones & Texto largo & Si & No & No & \\ \hline 
		InfiActores & Texto largo & Si & No & No & \\ \hline 	
	\end{tabular}
\end{table}


\begin{table}[ht]
	\caption{Tabla Informe Final}
	\label{labelTableInformeFinal}
	\begin{tabular}{ |l|l|l|l|l|l| }
		\hline
		Nombre & Tipo de dato & Valor Null & Primaria & For\'anea & Comentario \\ \hline
		InfiIdInformeFinal & Entero & Si & Si & No & \\ \hline 
		InfiProblemaPlanteado & Texto largo & Si & No & No & \\ \hline 
		InfiHipotesis & Texto largo & Si & No & No & \\ \hline 
		InfiAvanceObtenido & Texto largo & Si & No & No & \\ \hline 
		InfiRetosPlanteados & Texto largo & Si & No & No & \\ \hline 
		InfiGradoComprobacion & Texto largo & Si & No & No & \\ \hline 
		InfiGestionProyecto & Texto largo & Si & No & No & \\ \hline 
		InfiObservaciones & Texto largo & Si & No & No & \\ \hline 
		InfiRecomendaciones & Texto largo & Si & No & No & \\ \hline 
		InfiActores & Texto largo & Si & No & No & \\ \hline 	
	\end{tabular}
\end{table}


\begin{table}[ht]
	\caption{Tabla Cronograma Convocatoria}
	\label{labelTableCronogramaConvocatoria}
	\begin{tabular}{ |l|l|l|l|l|l| }
		\hline
		Nombre & Tipo de dato & Valor Null & Primaria & For\'anea & Comentario \\ \hline
		CrcvIdCronograma & Entero & Si & Si & No & \\ \hline 
		CrcvDescripcion & Texto largo & Si & No & No & \\ \hline 
		CrcvFechaCreacion & Fecha & Si & No & No & \\ \hline 
		CrcvFechaActividad & Fecha & Si & No & No & \\ \hline 
		CrcvEstado & Cadena & No & No & No & \\ \hline 
		ConvFKIdConvocatoria & Entero & Si & No & Si & \\ \hline 	
	\end{tabular}
\end{table}

\begin{table}[ht]
	\caption{Tabla Soportes Cronograma}
	\label{labelTableSoportesCronograma}
	\begin{tabular}{ |l|l|l|l|l|l| }
		\hline
		Nombre & Tipo de dato & Valor Null & Primaria & For\'anea & Comentario \\ \hline
		SocrIdSoporte & Entero & Si & Si & No & \\ \hline 
		SocrDescripcion & Texto largo & Si & No & No & \\ \hline 
		SocrRutaAdjunto & Cadena & Si & No & No & \\ \hline 
		SocrEstado & Cadena & Si & No & No & \\ \hline 
		SocrFecha & Fecha & Si & No & No & \\ \hline 
		CrcvFKIdCronograma & Entero & Si & No & Si &  \\ \hline	
	\end{tabular}
\end{table}

\begin{table}[ht]
	\caption{Tabla Producci\'on Grupo}
	\label{labelTableProduccionGrupo}
	\begin{tabular}{ |l|l|l|l|l|l| }
		\hline
		Nombre & Tipo de dato & Valor Null & Primaria & For\'anea & Comentario \\ \hline
		PgruIdProduccion & Entero & Si & Si & No & \\ \hline 
		Pgru & Cadena & No & No & No & \\ \hline 	
	\end{tabular}
\end{table}

\begin{table}[ht]
	\caption{Tabla Unidades Acad\'emicas}
	\label{labelTableUnidadesAcademicas}
	\begin{tabular}{ |l|l|l|l|l|l| }
		\hline
		Nombre & Tipo de dato & Valor Null & Primaria & For\'anea & Comentario \\ \hline		
		UnacIdUnidad & Entero & Si & Si & No & \\ \hline 
		UnacNombre & Cadena & Si & No & No & \\ \hline 
		UnacEstado & Cadena & Si & No & No & \\ \hline 
		UnacFechaCreacion & Fecha & Si & No & No & \\ \hline	
	\end{tabular}
\end{table}
	\chapter{Licitaci\'on de requisitos}

\begin{table}[ht]
	\centering
	\caption{Requisito - Gestionar convocatoria}
	\label{my-label}
	\begin{tabular}{|l|p{13cm}|}
        \hline
		OBJ-001 & \begin{tabular}[c]{@{}l@{}}
			Gestionar convocatorias
		\end{tabular} \\ \hline
		Versi\'on & 002 (2016-08-03) \\ \hline
		Autores & \begin{tabular}[c]{@{}l@{}}
			ANGIE ZULETA CARDONA (Universidad de la Amazonia)\\
			Derly Viviana  Murcia Serrano (Universidad de la Amazonia)\\
			Mateo Ceballos Bermudez (Universidad de la Amazonia) \\
			Nicol Dayana Endo Ruiz (Universidad de la Amazonia)
		\end{tabular} \\ \hline
		Fuentes & \begin{tabular}[c]{@{}l@{}}
			Alberto Fajardo Oliveros (Uniamazonia) \\
			Dora Lida Suarez Castro (Uniamazonia)
		\end{tabular}  \\ \hline
		Descripci\'on & \begin{tabular}[c]{@{}l@{}}
			El sistema deber\'a gestionar la informaci\'on \\ 
			respecto a las distintas convocatorias en el \\ 
			marco de semilleros y grupos de investigaci\'on.
		\end{tabular} \\ \hline
		SubObjetivos & \\ \hline
		Importancia & Vital \\ \hline
		Urgencia & Inmediatamente \\ \hline
		Estado & En Construcci\'on \\ \hline
		Estabilidad & Alta \\ \hline
		Comentarios & Ninguno \\ \hline
	\end{tabular}
\end{table}


\begin{table}[ht]
	\centering
	\caption{Requisito - Gestionar propuestas de proyectos}
	\label{my-label}
	\begin{tabular}{|l|l|l|l|l|}
        \hline
		OBJ-002      & \begin{tabular}[c]{@{}l@{}}
			Gestionar propuestas de proyectos
		\end{tabular} &  &  &  \\ \hline
		Versi\'on      & 003 (2016-08-03) &  &  &  \\ \hline
		Autores      & \begin{tabular}[c]{@{}l@{}}
			ANGIE ZULETA CARDONA (Universidad de la Amazonia)\\
			Derly Viviana  Murcia Serrano (Universidad de la Amazonia) \\ 
			Mateo Ceballos Bermudez (Universidad de la Amazonia) \\ 
			Nicol Dayana Endo Ruiz (Universidad de\\   la Amazonia)
		\end{tabular} &  &  &  \\ \hline
		Fuentes      & \begin{tabular}[c]{@{}l@{}}
			Alberto \\   Fajardo Oliveros (Uniamazonia) Dora Lida Suarez Castro (Uniamazonia)
		\end{tabular} &  &  &  \\
		Descripci\'on  & \begin{tabular}[c]{@{}l@{}}
			El sistema debe gestionar la informaci\'on que\\   respecta a cada una de las propuestas que se presentan, sin importar el\\   origen de estas.\end{tabular} &  &  &  \\  \hline
		SubObjetivos & & & &  \\  \hline
		Importancia  & Vital &  &  &  \\ \hline
		Urgencia     & Inmediatamente &  &  &  \\
		Estado       & En Construcci\'on &  &  &  \\
		Estabilidad  & Alta &  &  &  \\ \hline
		Comentarios  & Ninguno &  &  & \\  \hline
	\end{tabular}
\end{table}




	
	% Finalizaci�n
	%
%\makeglossaries
%\loadglsentries{glossaires}
%
%\newglossaryentry{latex}
%{
%	name=latex,
%	description={Is a mark up language specially suited 
%		for scientific documents}
%}
%
%\newglossaryentry{maths}
%{
%	name=mathematics,
%	description={Mathematics is what mathematicians do}
%}

%\title{How to create a glossary}
%\author{ }
%\date{ }

%\maketitle

%The \Gls{latex} typesetting markup language is specially suitable 
%for documents that include \gls{maths}. 

%\clearpage

%\printglossaries
	\include{capitulos/conclusiones}
	
	\bibliographystyle{alpha}
	\bibliography{referencias}
	
	\section*{Anexos}


\begin{figure}[htbp]
	\centering
	\includegraphics[width=100mm]{resources/capturas/inicio.png} 
	\caption{Pagina de inicio.} \label{anexo:pagina_inicio1}

	\includegraphics[width=100mm]{resources/capturas/inicio2.png}
	\caption{Pagina de inicio.} \label{anexo:pagina_inicio2}
\end{figure}
\end{document}