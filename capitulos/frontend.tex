\chapter{Arquitectura de lado del cliente}
\label{chapter_arq_frontend}

Con el auge de las plataformas web y la necesidad de hacer presencia en la red de Internet, el desarrollo de servicios web multi-plataforma se convierte en un requisito fundamental que cumple con la demanda generada por las personas.
\\

Microsoft dispone de muchas tecnolog�as, van desde sistemas operativos (SO) hasta peque�as pero potentes librer�as que facilitan el desarrollo de software. El framework .NET creado por esta multinacional, destaca el manejo r�pido y econ�mico de desarrollar aplicaciones permitiendo una integraci�n m�s r�pida y eficaz a diferentes dispositivos, actualmente disponible para su instalaci�n en SO Windows, Linux y Mac, aunque estos dos �ltimos disponen de una versi�n comunitaria llamada .NET Core que incluye el acceso a la mayor�a de las APIs dadas por su gemelo .Net, por otro lado, cabe destacar que .Net Core pertenece a una rama diferente al compilador Mono. 
\\

Este proyecto se ha realizado con base en esta tecnolog�a (.NET), desarrollando una aplicaci�n web completa para gesti�n documental de los procesos dados por administraci�n y vicerector�a.de la Universidad de la Amazonia.

\section{Herramientas}

\begin{table}[H]
	\centering
	\caption{Herramientas usadas en las interfaces cliente}
	\label{label_herr_impl}
	\begin{tabularx}{\linewidth}{|l|X|l|}	
		\hline 
		\textbf{Nombre} & \textbf{Descripci�n} & \textbf{Versi�n} \\ 
		\hline 
		Ext.NET	& Framework de componentes basado en ASP.NET, construido por Sencha. Multinacionales de todo el mundo lo usan: Disney, Canon, Universal, entre otras. EL men� principal de SIGEPI junto con todas sus opciones fue desarrollado con este marco de trabajo con el fin de mejorar la experiencia de usuario y la usabilidad del sistema. & 4 \\ 
		\hline 
		Bootstrap & Framework HTML, CSS y JS popular, usado para desarrollar peque�as animaciones y colores en la pagina principal. & 3 \\ 
		\hline
		AngularJS & Framework de javascript desarrollado por Google que permite el desarrollo de interfaces web din�micas. Usado en la pagina principal para mejorar la interacci�n con el cliente. & 1 \\ 
		\hline 
		jQuery & Biblioteca de JavaScript, simplifica la manera de interactuar con documentos HTML, �rbol DOM, eventos, animaciones y AJAX. Usado en el dise�o de la pagina principal de SIGEPI & 1 \\
		\hline
	\end{tabularx} 
\end{table}