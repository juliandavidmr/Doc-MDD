\chapter{Introducci�n}

Actualmente, adaptarse a las necesidades del cliente resulta algunas 
veces en un problema que aumenta con el transcurso del tiempo, las 
reglas de negocio cambian y consigo tambi�n la construcci�n de sistemas
de informaci�n, de esta manera, el mundo moderno exige gran demanda tecnol�gica
y adaptarse a esta linea de constantes transformaciones resulta
en una tarea compleja. Por esto, gracias a varias investigaciones
realizadas en este campo, se han logrado avances en ciertas �reas
fundamentales del desarrollo de software que influyen en la puesta 
en marcha de proyectos.
\\

Muchos proyectos de tipo software est�n siendo solicitados por entidades
para satisfacer m�ltiples prop�sitos. La demanda es alta cuando se 
usan buenas practicas de desarrollo junto con herramientas orientadas al
rendimiento y eficiencia en el cumplimiento de tareas.
Los tiempos de ejecuci�n y elaboraci�n de proyectos se
reducen cuando herramientas DSL (Lenguaje de dominio especifico)
son implementadas, ofreciendo caracter�sticas de generaci�n de
c�digo fuente reutilizable y/o componentes modulares.
\\

El lenguaje de dominio especifico no es una tecnolog�a emergente,
las primeras apariciones fueron a mediados de los 80s. 
Actualmente existen derivaciones de DSL para diferentes aplicaciones,
cada una dise�ada para la ejecuci�n de funcionalidades especificas, 
tales como MDD (Desarrollo Dirigido por Modelos), MDA (Arquitectura Dirigida 
por Modelos) y MDE (Ingenier�a Orientada a Modelos), siendo estos 
un conjunto de recursos interoperables que permiten ser usados en el
an�lisis, modelado y construcci�n de proyectos software.
Por otro lado, han aumentado la cantidad de mecanismos para agilizar
realizaci�n de proyectos, generalmente estos siguen la misma filosof�a 
de las DSL, por ejemplo, se pueden encontrar las ORM (Mapeo de Objetos 
Relacionales) para el control de m�ltiples gestores de bases de datos desde
un solo lenguaje base.
Del mismo modo, los ingenieros, programadores de computadoras, 
matem�ticos, estad�sticos y dem�s labores afines, hacen uso de herramientas 
DSL a partir de programas de computadora 
y lenguajes de programaci�n tales como R (Lenguaje estad�stico), 
interpretes de expresiones regulares, SQL (lenguaje de consulta estructurada), 
entre otros. Como se puede observar, un dominio espec�fico 
tiene un vocabulario especializado para describir las cosas que son 
particulares a ese dominio.
\\

El sistema de gesti�n documental de proyectos de investigaci�n, SIGEPI, 
es una plataforma web robusta, creada gracias a la implementaci�n de 
herramientas especializadas en el modelado y generaci�n de c�digo fuente. 
En la primera secci�n de este documento se nombran todos aquellos conceptos 
que detallan los procesos que se siguieron para la obtenci�n del c�digo 
fuente a partir de los esquemas dados al aplicar MDD. Posteriormente, se 
explica la arquitectura de desarrollo que fue usada para la plataforma web, 
incluyendo cada una de las librer�as y marcos de trabajo que fueron usados.