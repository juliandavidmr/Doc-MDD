\section[INTRODUCCI�N]{INTRODUCCI�N}

Hoy en d�a, muchos proyectos de desarrollo de
aplicaciones y sistemas inform�ticos poseen gran
demanda, y m�s cuando est�n basados en
herramientas de automatizaci�n de tareas.
Los tiempos de ejecuci�n y elaboraci�n de 
proyectos se reducen cuando herramientas DSL (Lenguaje de dominio 
especifico) son implementadas, ofreciendo caracter�sticas 
de generaci�n de c�digo reutilizable, tales como componentes que 
pueden ser implementados en otros proyectos del mismo tipo. No se 
trata de una tecnolog�a emergente, las primeras apariciones de 
los DSL fueron a mediados de los 80's.
\\

Actualmente se pueden encontrar ramificaciones DSL 
en diferentes aplicaciones, cada una con
funcionalidades especificas, tales como MDD, MDA y
MDE, siendo estos un conjunto de recursos
interoperables que permiten ser usados en el
an�lisis, modelado y construcci�n de proyectos
software. Por otro lado, las nuevas tecnolog�as han
aumentado la cantidad de mecanismos y maneras de
generar software, generalmente todos estos siguen la
misma filosof�a de las DSL, por ejemplo, se pueden
encontrar las ORM (Mapeo de Objetos Relacionales) para el
control de m�ltiples gestores
de bases de datos a partir de un solo lenguaje base. Del mismo 
modo, los programadores de software, matem�ticos, estad�sticos y 
dem�s, hacen uso de herramientas de lenguaje de dominio 
especifico a partir de programas de computadora como R, interpretes 
de expresiones regulares, SQL (lenguaje de consulta estructurada), entre otros.
\\

El sistema de gesti�n documental, SIGEPI, es una plataforma web 
robusta, creada gracias a la implementaci�n de herramientas 
de modelado y generaci�n de c�digo, DSL. En la primera secci�n de 
este documento se nombran todos aquellos procesos que se siguieron 
para la obtenci�n del c�digo fuente a partir de los esquemas 
dados al aplicar MDD (Desarrollo Dirigido por Modelos).
Posteriormente, se explica en detalle la arquitectura de desarrollo que fue
aplicada, incluyendo cada una de las librer�as y marcos de trabajo que 
fueron usados.