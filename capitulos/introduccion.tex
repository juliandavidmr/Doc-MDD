\section[INTRODUCCI�N]{INTRODUCCI�N}

Actualmente, adaptarse a las necesidades del cliente es un
problema que aumenta con el transcurso del tiempo, las condiciones
de negocio cambian y consigo tambi�n la construcci�n de sistemas.
Consecutivamente, el mundo moderno exige gran demanda tecnol�gica
y adaptarse a esta linea de constantes transformaciones resulta
en una tarea compleja. Por esto, gracias a las investigaciones
realizadas en este campo, se han logrado avances en ciertas �reas 
fundamentales que influyen en la puesta en marcha de proyectos.
\\

Muchos proyectos de desarrollo de software poseen gran demanda, y
m�s cuando est�n soportados por herramientas de automatizaci�n
de tareas. Los tiempos de ejecuci�n y elaboraci�n de proyectos se
reducen cuando herramientas DSL (Lenguaje de dominio especifico)
son implementadas, ofreciendo caracter�sticas de generaci�n de
c�digo reutilizable y componentes.
\\

El lenguaje de dominio especifico no es una tecnolog�a emergente,
las primeras apariciones fueron a mediados de los 80s. 
Actualmente se pueden encontrar ramificaciones DSL 
en diferentes aplicaciones, cada una con
funcionalidades especificas, tales como MDD (Desarrollo Dirigido
por Modelos), MDA (Arquitectura Dirigida por Modelos) y
MDE (Ingenieria Orientada a Modelos), siendo estos un conjunto de
recursos interoperables que permiten ser usados en el
an�lisis, modelado y construcci�n de proyectos
software. Por otro lado, han aumentado la cantidad de mecanismos
para agilizar la planeaci�n y ejecuci�n de proyectos, generalmente todos estos siguen la misma filosof�a de las DSL, por ejemplo, se
pueden encontrar las ORM (Mapeo de Objetos Relacionales) para el
control de m�ltiples gestores
de bases de datos a partir de un solo lenguaje base. Del mismo 
modo, los ingenieros, programadores de computadoras, matem�ticos,
estad�sticos y dem�s, hacen uso de herramientas de lenguaje de
dominio 
especifico a partir de programas de computador como R (Lenguaje
estad�stico), interpretes 
de expresiones regulares, SQL (lenguaje de consulta estructurada),
entre otros.
\\

El sistema de gesti�n documental, SIGEPI, es una plataforma web 
robusta, creada gracias a la implementaci�n de herramientas 
de modelado y generaci�n de c�digo, DSL. En la primera secci�n de 
este documento se nombran todos aquellos procesos que se siguieron 
para la obtenci�n del c�digo fuente a partir de los esquemas 
dados al aplicar MDD (Desarrollo Dirigido por Modelos).
Posteriormente, se explica en detalle la arquitectura de desarrollo que fue
aplicada, incluyendo cada una de las librer�as y marcos de trabajo que 
fueron usados.