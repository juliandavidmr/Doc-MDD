\section[INTRODUCCI�N]{INTRODUCCI�N}

Hoy en d�a, muchos de los desarrollos de aplicaciones y sistemas inform�ticos con funcionalidades para automatizaci�n de tareas poseen gran demanda. Los tiempos de ejecuci�n de un proyecto se reducen cuando herramientas DSL (Lenguaje de dominio especifico) y similares son implementadas, ofreciendo caracter�sticas de generaci�n de c�digo reusable, tales como componentes que pueden ser implementados en otros proyectos del mismo tipo. No se trata de una tecnolog�a emergente, las primeras apariciones del est�ndar DSL fueron lanzadas a principios del a�o XXXX.
\\

Actualmente se pueden encontrar ramificaciones DSL en varios grupos, tales como las MDD, MDA, MDE, DSM, ORM, siendo estos unos de los mas conocidos.