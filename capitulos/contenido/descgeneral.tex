\section[DESCRIPCI�N GENERAL DEL PROYECTO]{DESCRIPCI�N GENERAL DEL PROYECTO}

\subsection[Planteamiento del problema]{Planteamiento del problema}

\subsubsection[Contexto]{Contexto}
En la b�squeda de mejores procesos que ayuden a la optimizaci�n y mejoramiento de la productividad, nuevas metodolog�as y herramientas han emergido, consigo vienen toda clase de formas para realizar tareas complejas en tiempos relativamente cortos. Un ejemplo claro es el modelado de requisitos, siguiendo el est�ndar UML se pueden obtener varios esquemas que permiten la visualizaci�n de cada proceso por separado desde diferentes puntos de vista en un sistema; casos de uso, clases, bloques, secuencia, etc.

Es cierto que el est�ndar UML permite la generaci�n de c�digo fuente, pero existe un problema que se puede plantear sobre la misma ideolog�a que mantiene este est�ndar, dado que se deben seguir estrictamente una serie de normas. Algunas veces es necesario dise�ar y crear un sistema siguiendo un conjunto de normas no dadas por un est�ndar existente, propiamente personalizadas y generalmente creadas desde cero a partir de una base abstracta obtenida desde los requisitos, es decir, crear una serie de reglas de modelado a partir de un grupo de requisitos. Aqu� es donde entran en acci�n las DSL, permitiendo el establecimiento de un conjunto de reglas para el modelado de esquemas personalizados siguiendo como base fundamental los requisitos del sistema. De esta manera, cada proceso que se propone para la construcci�n del sistema va a seguir estrictamente ese un conjunto de reglas individualizadas.

\subsubsection[Formulaci�n del problema]{Formulaci�n del problema}

\subsection[Justificaci�n]{Justificaci�n}


\subsection[Objetivos]{Objetivos}
\subsubsection[Objetivo general]{Objetivo general}
Implementar el sistema de informaci�n para la gesti�n de los procesos de presentaci�n, evaluaci�n y seguimiento de los proyectos de investigaci�n en la Universidad de la Amazonia, mediante el tratamiento y aplicaci�n de los resultados obtenidos en el uso de la arquitectura de desarrollo dirigida por modelos, MDD.

\subsubsection[Objetivo especifico]{Objetivo especifico}
\begin{itemize}
	\item Procesar y generar resultados (c�digo fuente) a partir de los meta-modelos definidos por los esquemas dise�ados en la arquitectura MDD.
	\item Aplicar el c�digo fuente obtenido por el procesamiento de los meta-modelos en un entorno de desarrollo basado en la web, espec�ficamente la arquitectura de tres capas con .NET MVC.
	\item Realizar pruebas de rendimiento, usabilidad y seguridad para verificar el cumplimiento de los requisitos no funcionales referentes al mismo.
\end{itemize}
