\chapter[Marco referencial]{Marco referencial}

\section[Marco te�rico]{Marco te�rico}

\subsection[Desarrollo Dirigido por Modelos. Conceptos]{Desarrollo Dirigido por Modelos. Conceptos}
El desarrollo dirigido por modelos (MDD) es un paradigma que resuelve
inconvenientes relevantes en el desarrollo de software, la causa 
surgi� desde los inicios de la d�cada de los 60s, cuando se
introdujo el concepto de ''crisis del software'', originado por 
la complejidad y el costo requerido por las necesidades del cliente \cite{pons2010desarrollo}.
MDD enriquece el proceso de desarrollo a partir de herramientas 
especializadas, destaca una relaci�n entre lo abstracto y el c�digo
fuente. La abstracci�n est� representada por modelos, por tanto 
estos requieren de una serie de fases culminar la transformaci�n hasta
obtener el c�digo fuente [ver figura \ref{fig_transcode}].

\begin{displaymath}
	\xymatrix{
		*+<1cm>[F-,]{Modelo}\ar[r] 
		& *+<1cm>[o][F]{M2T/M2M}\ar[r]
		& ... \ar[r]
		& *+<1cm>[F-,]{Modelo}\ar[r]
		& *+<.5cm>[F]{Codigo}
	}
\end{displaymath}
\captionof{figure}{Proceso de transformaci�n de c�digo}
\label{fig_transcode}

\subsection[Lenguaje de dominio especifico]{Lenguaje de dominio especifico}
El lenguaje de dominio especifico (DSL) eleva el nivel de abstracci�n m�s all�
que los lenguajes de programaci�n para especificar una soluci�n al
problema usando conceptos de dominio \cite{kelly2008domain}. 
Generalmente se le encuentra como una notaci�n gr�fica, cuyos
modelos resultan en un conjunto de elementos y relaciones entre si.

\begin{displaymath}
	\xymatrix{
		\underline{Requerimientos}\ar@/_1mm/[r] 
		& *+<1cm>[o][F]{\texttt{Capa de abstracci�n}}\ar[r]
		& *+<1cm>[F-,]{\textbf{DSL}}
	}
\end{displaymath}
\captionof{figure}{Flujo de abstracci�n DSL}
\label{fig_abs_dsl}

\subsection[Arquitectura de desarrollo]{Arquitectura de desarrollo}