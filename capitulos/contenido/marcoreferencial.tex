\chapter[Marco referencial]{Marco referencial}

\section[Marco te�rico]{Marco te�rico}

\subsection[Desarrollo Dirigido por Modelos. Conceptos]{Desarrollo Dirigido por Modelos. Conceptos}
El desarrollo dirigido por modelos (MDD) es un paradigma que resuelve
inconvenientes relevantes en el desarrollo de software, la causa 
surgi� desde los inicios de la d�cada de los 60s, cuando se
introdujo el concepto de ''crisis del software'', originado por 
la complejidad y el costo requerido por las necesidades del cliente \cite{pons2010desarrollo}.
MDD enriquece el proceso de desarrollo a partir de herramientas 
especializadas, destaca una relaci�n entre lo abstracto y el c�digo
fuente. La abstracci�n est� representada por modelos, por tanto 
estos requieren de una serie de fases culminar la transformaci�n hasta
obtener el c�digo fuente [ver figura \ref{fig_transcode}].

\begin{displaymath}
	\xymatrix{
		*+<1cm>[F-,]{Modelo}\ar[r] 
		& *+<1cm>[o][F]{M2T/M2M}\ar[r]
		& ... \ar[r]
		& *+<1cm>[F-,]{Modelo}\ar[r]
		& *+<.5cm>[F]{Codigo}
	}
\end{displaymath}
\captionof{figure}{Proceso de transformaci�n de c�digo}
\label{fig_transcode}

\subsection[Lenguaje de dominio especifico]{Lenguaje de dominio especifico}
El lenguaje de dominio especifico (DSL) eleva el nivel de abstracci�n m�s all�
que los lenguajes de programaci�n para especificar una soluci�n al
problema usando conceptos de dominio \cite{kelly2008domain}. 
Generalmente se le encuentra como una notaci�n gr�fica, cuyos
modelos resultan en un conjunto de elementos y relaciones entre si.

\begin{displaymath}
	\xymatrix{
		\underline{Requerimientos}\ar@/_1mm/[r] 
		& *+<1cm>[o][F]{\texttt{Capa de abstracci�n}}\ar[r]
		& *+<1cm>[F-,]{\textbf{DSL}}
	}
\end{displaymath}
\captionof{figure}{Flujo de abstracci�n DSL}
\label{fig_abs_dsl}

\subsection{Arquitectura de desarrollo}

La arquitectura de desarrollo aplicada en este proyecto 
consiste de un m�todo de construcci�n de sistemas interactivos,
esta forma ofrece al usuario programador un conjunto completo 
de herramientas de trabajo en paralelo mediante m�ltiples 
canales e interfaces de usuario \cite{lucassen2006mvc}. En el 
contexto actual, este patr�n de arquitectura MVC
(Modelo, Vista y Controlador)\cite{leff2001web} 
ha sido implementado con base principal en las tecnolog�as de
desarrollo dadas por la compa��a estadounidense Microsoft, 
[ver figura \ref{arq_des_mvc}].

\begin{center}
	\begin{tikzpicture}[node distance=1cm, auto,]
		%nodes
		\node[punkt] (market) {Modelos (M)};
		\node[punkt, inner sep=5pt,below=0.5cm of market]
		(formidler) {Controladores (C)};
		% We make a dummy figure to make everything look nice.
		\node[above=of market] (dummy) {};
		\node[right=of dummy] (t) {APIs del Sistema}
		edge[pil,bend left=45] (market.east) % edges are used to connect two nodes
		edge[pil, bend left=45] (formidler.east); % .east since we want
		% consistent style
		\node[left=of dummy] (g) {Intercambio de datos}
		edge[pil, bend right=45] (market.west)
		edge[pil, bend right=45] (formidler.west)
		edge[pil,<->, bend left=45] node[auto] {Vistas (V)} (t);
	\end{tikzpicture}
\end{center}
\vspace{1em}

\captionof{figure}{Arquitectura de desarrollo MVC}
\label{arq_des_mvc}